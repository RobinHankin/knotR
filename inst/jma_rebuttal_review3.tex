\documentclass[12pt]{article}
\usepackage{xcolor}
\begin{document}

(Below, the reviewer's comments are in black, and my replies to the
issues are in \textcolor{blue}{blue}.  I have indicated changes to the
manuscript where appropriate.

\section*{Rebuttal to second third review: overall response}

Short story: the reviewers make reasonable and constructivde comments.
I have accommodated the comments with  with rewording.  The resulting
document is, I believe, stronger and more scholarly than before and I
recommend it to you.


\section*{Detailed rebuttal: Reviewer \#1}


Comments from the Editors and Reviewers:

Thank you for this revised manuscript.  My apologies for contributing
to the slow turnaround of these reviews.

I have given this version the official status of ``Minor Revision" to
reflect the fact that we all agree that this draft is much closer to
publication-ready. In particular, I believe that the remaining
reviewer concerns can be addressed with moderate rewriting.  However,
some of the reviewer feedback falls closer to ``Major Revision," so I
wanted to acknowledge that a few of the remaining critiques are beyond
surface level.

As you will see in the detailed feedback below, one of the reviewers
still has substantive concerns about the overall framing of your
discussion and how directly it addresses the mathematical and
aesthetic choices that went into the software package.  I suspect that
you can address the concerns in the first few review paragraphs by
being more specific about the constraints of your implementation, the
level of flexibility you hoped to provide for the user, and the
alternate possibilities for future work, but since I don't know what
technical decisions went into your coding, I might be misjudging
exactly what is suitable.

I look forward to receiving your further revision, and will make sure
the editorial process goes more quickly from this point forward.

Best wishes,

Susan Goldstine, Ph.D.
Associate Editor
Journal of Mathematics and the Arts


Reviewer \#1: The paper has improved in terms of its overall
presentation in the context of knot renderings, and many detailed
fixes have been made as suggested by the reviewers.

However, the paper still has the major weakness of being too much a
somewhat ad-hoc technical manual on ``How-to-use-R and Inkplot to get
some decent knot plots" by using the software package [27].


The paper is lacking good discussions of the key issues and
explanations of the choices made, specifically: What makes a knot plot
attractive, followed by a thorough discussion of how to best set up a
knot representation and an optimization procedure to achieve the
desired qualities?

\textcolor{blue}{I appreciate the sentiment, which I take to be a
  comment on the large amount of programming-oriented detail in the
  submission.  This reviewer evidently regards the knots in the
  submission as ``decent'', which I take to be a great compliment.  It
  is often difficult to know the correct balance between overly
  technical descriptions [which indeed often overlaps with the aims of
    a technical manual] and gushing conceptual big-picture thinking.
  However, I note that many of the comments below are essentially
  requests for clarifications of minor technical details of the
  implementation: and this would indicate that perhaps the balance
  should be shifted back toward the technical manual end of the
  spectrum.  The other observation I would make here is that the
  software package [27] is intended to be part of the submission (it
  is released under the GPL and is thus available for public use) and
  with this understanding, even a downbeat technical manual would have
  scholarly value in conjunction with the new software it describes.
  I want the editor and reviewer to know that in addressing the
  concerns below I have tried hard to answer not just ``what'' but
  also the ``why'' of the implementation.  \\ \\ Nevertheless, the cue
  above is a good one.  The two questions: (a) {\em what makes a knot
    plot attractive?}, and (b) {\em how best ... to achieve this?}
  are indeed profound, and were not addressed directly in the old
  manuscript.  In the revision I have prominently stated the questions
  in the introduction, and given a provocative, if brief, outline of
  my view.}


The problem starts with the basic curve representation:

Why use Bezier curves? 

Why not use some other spline curve that guarantees G2-continuity?

And, if forced by the chosen tool to use Bezier curves, why not place
nodes at the crossing points, (and one or two more for large loopy
lobes between subsequent crossing points)?  -- This would allow the
user to place the crossing points far enough apart to start with, and
this would make it easy to set the Bezier handles to force right-angle
crossings.

Also, the implementation in the ``badness function" seems far from
optimal.  ``Total crossing angles()" does not address a key problem
directly: If one wants to avoid highly acute crossing angles, then
this can be addressed in a much more sensitive way with a function
such as 1/ sqr( cross.angle ).  This would go to infinity in the worst
case!  It also corresponds more directly to the way that excessive
curvature is avoided by minimizing ``bending energy," which is the
integral over the square of curvature -- or: 1/ sqr( curve-radius).
These are some ``math" issues that readers of JMA would be interested
in.


Some more detailed comments:

Page 3 line 42: ``...knots possess a line of symmetry (at least, the
diagrams do if the breaks are ignored), which ..."  -$>>$ ``...knots
possess a C2 rotation axis, which ..."

Page 4 lines 22-25: ``One plausible technique for creating knot
projections is to consider a two-dimensional projection of a knot's
embedding E in R3. However, this approach often results in poor visual
appearance: cusps or other displeasing effects can occur."  -$>>$ True!
- But what is the alternative, if the user has an entangled physical
knot in her hands?

Page 4 line 45: Define ``nodes" !

\textcolor{blue}{``Nodes'' is a standard word in the context of Bezier
  curves. I have not used the word in any other sense in the
  manuscript and I think the text is clear as it stands.}

Page 4 line 48: ``However, if we can ensure that the radius of
curvature changes only by a small amount,..."  -$>>$ HOW can the user do
this?

Page 4 lines 50-52: No need to differentiate here between over- and
under-strands.  I assume they are treated the same in the optimization
process.

\textcolor{blue}{The referee is correct to say that the over and under
  strand are treated the same by the optimization routine, but my
  intent was to introduce and define the words themselves (rather than
  to differentiate between them).}

Page 4 lines 53-54: ``We would like strand crossing points to be far
from nodes,..."  -- Why?  The curves are also visually continuous across
the nodes!

\textcolor{blue}{The curves certainly are visually continuous across
  the nodes---but only in the optimized knots.  The motivation for
  keeping Bezier nodes far from crossing points is twofold.  Firstly,
  as one traces a curve visually, Bezier nodes cannot possibly avoid
  introducing {\em some} form of disjointness in the path; and such
  disjointness is particularly disruptive if on the understrand of a
  node, as the visual continuity is interrupted.  I have added a
  figure (including two different diagrams of a trefoil knot) to the
  Gallery that illustrates the reason why nodes' being at crossing
  points is undesirable.  In short, at the node of a Bezier curve, the
  radius of curvature undergoes a visually jarring discontinuity which
  looks bad, especially the understrand, which is particularly
  susecptible to such considerations because of the break.  I do
  appreciate that many viewers will find the distinction invisible or
  inconsequential; but my informal polling of family and colleagues
  reveals that, once sensitised to the issue, they cannot unsee the
  kink.  \\ \\ The second reason is more mundane.  With reference to
  figure 7 (the four figure-of-eight knots diagram), we can see a
  vertical line of symmetry on which there are two crossing points.
  Imagine there was a node at one of these crossing points.  There
  would have to be {\em two} nodes at the same point, one for the
  overstrand and one for the understrand; the software would somehow
  have to impose vertical mirror symmetry on the diagram and this
  would necessitate that the knot class contain yet another symmetry
  object, alongside the 12 existing ones.  These considerations become
  yet more onerous when considering rotational symmetry as in Figure 8
  (which shows $5_1$).  If there were a node at, say, the intersection
  of strand 18-19 and 9-10 then somehow the two nodes would have to
  embody not only mirror symmetry (with respect to 5-6/14-15) but also
  fivefold rotational symmetry.  Simply put, this is a nightmare.  The
  software documentation contains a substantial discussion of the
  issue, but I do not believe that the manuscript would benefit from
  what amounts to little more than me grumbling about implementation
  difficulties.}

Page 5 lines 46-51: Why are there SEVEN significant decimal digits?
Does the graphic screen have 10 million pixels per line?

\textcolor{blue}{This is the default print method used by R (which
  internally uses IEC-60559 arithmetic, 14 significant figures).  Any
  attempt by me to change the print method would, I believe, be
  confusing and unnecessary.  The point of the table was to show that
  a knot path could be represented by a few dozen floating-point
  numbers.}

Figure 3, 5, 7: Perhaps fewer than 100 circles per Bezier curve would
be better.


Figure 4.  Knot 76 with strands numbered... -- The main use here seems
to be to see very clearly the extent of every Bezier curve.
Over-/under-crossings seem to become relevant only for the final
rendering of the knot curve with the gaps included.

Page 7 line 52: ``The optimization typically proceeds over R 50" - What
is the meaning of this?

Page 7 line 56: ``...function total crossing angles()" The function is
not explicitly defined in the paper, but it sounds clearly suboptimal.
One bad acute-angle crossing can easily ```hide" in a sum of all angles.
It would be better to individually penalize acute angles.  A suitable
term might be: 1/ sqr( cross.angle ).  This would go to infinity in
the worst case!

Page 8 line 20, and footnote: The ``devil" and the ``interesting math"
is in the details!  The footnote is not sufficient to explain the
trade-offs that are being made in the optimization process.

Page 9, line 62: What is ``Bosch-type symmetry:..." ?  Please explain.


\textcolor{blue}{Bosch-type symmetry refers to Bosch (2010) which was
  cited in the previous sentence.\\ \\ old text: The package
  implements Bosch-type symmetry\ldots \\ \\ new text: The package
  implements symmetry in a similar way to that of Bosch (2010)\ldots}
  
$>>$ If all nodes fulfill the specified symmetry constraints (as shown
on page 11, lines 15-22 and on lines 35-43), but the connection
diagram does not completely adhere to the specified symmetry - What
happens? - Does the package check this and let the user know about the
discrepancy?

Page 10 lines 56-58: ``Minimizing the badness is not entirely
straightforward..."  -- What is the problem?  I think that if all the
points are moved jointly within the specified symmetry constraints, an
optimized symmetrical solution will result.

{The above two comments suggest, that it might be advantageous to use
  some hierarchy in the representation of symmetrical knots: Only
  specify the curve for one unique fragment of the knot curve, and
  compose the complete curve by instancing that fragment with
  mirroring and rotation as required to construct the complete knot!
  The optimization would then only have to be done for this unique
  fragment (with proper end-conditions).  If this same hierarchy is
  also used in the graphical display of the knot curve, the user could
  easily make adjustments to just a few nodes, while always
  automatically obtaining the desired symmetrical result.}

Page 10 lines 55: ``In the package, the badness weightings may be
altered easily,..."  -- What is the user interface for this?  Are
there sliders in the display that allow interactive adjustments of the
weights?

Figure 9: The fact that ``angle-crossing penalty" had to be enhanced by
a factor of 100 to achieve a more desirable (in my opinion) result,
illustrates the fact that the "total crossing angles()"-function is
not a good choice for optimization.

I am missing a description how the two breaks at every crossing point
are being graphically implemented.

Conclusions:

RE: ``Further work might include functionality to deal
with links."  $>>$ Readers may think that this would be a trivial
extension.  Please add a paragraph like the one you wrote in response
to Reviewer 1, to explain why the extension to links may be more
difficult than one might think.

RE: ``In the broader context of optimization in art, we observe that
numerical optimization techniques can produce aesthetically pleasing
results, an observation that might find uptake by graphic artists."
$>>$ This only holds with the proviso that we can think of a good
mathematical formulations to capture the various features that capture
aesthetics appropriately.  - This then becomes the real ``art."

Gallery:

Knot diagrams 83 through 86 in Figure 12 stand out with their helices
too tightly wound.  The two criteria on page 4, lines 54 and 55 do not
seem to be sufficiently enforced.  I suggest adjusting these figures
in the spirit of Figure 9b.  This may happen automatically with a
better ``badness function" that penalizes non-orthogonal crossing
angles more directly.

[27]:  Incomplete reference.  Perhaps give a URL.

\textcolor{blue}{Canonical URL now added}


Reviewer \#2: Firstly I would like to apologise for my late response on
this. Reviewing this edited paper had not got onto my radar.

Secondly all the points in my original report have now been
addressed. I do feel that the literature review could have gone a
little bit further beyond my stated suggestions, but it is now
sufficient to give the paper more context.

\end{document}
