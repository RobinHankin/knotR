\documentclass[12pt]{article}
\usepackage{xcolor}
\begin{document}

\section*{Rebuttal to second editorial review: overall response}

(Below, the reviewer's comments are in black, and my replies to the
issues are in \textcolor{blue}{blue}.  I have indicated changes to the
manuscript where appropriate.

Short story: I have accommodated all the comments with rewording.  The
resulting document is, I believe, stronger and more scholarly than
before and I recommend it to you.  


\section*{Detailed rebuttal: Reviewer \#1}

This short article is a tutorial example of how to use Inkscape and
KnotR to beautify 2-dimensional knot diagrams found in the knot
tables.  The cost function used in optimizing a given diagram tries to
minimize the overall bending energy of the knot curve, preserve all
shown 2D symmetries, and render all strand crossings as close to 90
degrees as possible.  The paper does not discuss the underlying
mathematics by which this optimization is achieved, but relies on the
power built into the software packages mentioned.  Also, the user has
to characterize explicitly the desired symmetries; there is no ``magic"
power that discovers the maximal possible symmetry that a given knot
can assume.  This exemplified by Figure 7, where the Figure-8-knot is
drawn with a single mirror axis, even though this same knot can
readily be drawn with two mirror lines; -- and also in Figure 9 where
a symmetrical and an asymmetrical rendering of the same knot is
presented.

\textcolor{blue}{This is a good characterisation of the submission,
  but the comment that there is no ``magic" power ascertaining
  maximal possible symmetry that a given knot may assume is a good
  one, and one that I had not explicitly realised.  I have added a
  brief discussion of this issue to the caption of Figure 9 that
  discusses the Perko pair:
  \begin{description}
    \item[OLD text] ``Two representations of knot~$10_{125}$, known as the 
     Perko Pair"
    \item[NEW text] ``Two representations of knot~$10_{125}$, known as the 
      Perko Pair.   The software requires the user to specify the symmetry
      (mirror or rotational) of a knot projecton and has no notion of
      topological invariance of a knot"
      \end{description}
}

This paper is clearly useful to someone who specifically wants to
clean up a given rendering of a particular knot, but it offers little
to readers who are not familiar with Inkscape and KnotR.

\textcolor{blue}{In any software, one makes use of previously written
  work.  One rarely writes an operating system from scratch, for
  example.  The knotR package leverages the design capabilities of
  inkscape in, I think, a useful and informative way; and it further
  uses the numerical abilities of R in an efficient and natural way.}%issue17


Detailed comments about the presentation:

Abstract: It should mention the software packages that are used in
this paper.

\textcolor{blue}{Done.  The abstract is now completely rewritten.}

P1, line 46:  ``Knot theory" -- ?  -- Is this an incomplete reference?

\textcolor{blue}{Corrected, this was a typo}

P2, line 43: ``uniform" curvature is not possible, since many knots
have inflection points.  It is better to ask for 'smoothly changing'
curvature, and to limit maximal curvature.

\textcolor{blue}{Corrected:
  \begin{description}
  \item[OLD text] Curvature to be as uniform as possible
\item[NEW text] Curvature to be as smoothly changing as possible, with
  limited maximal curvature
  \end{description}
}
  


P2, line 46: What does it mean for symmetry to be ``present" in the
knot?

\textcolor{blue}{
  \begin{description}
  \item[OLD text] Any symmetry present in the knot should be enforced exactly, and be visually apparent
\item[NEW text] Any symmetry desired in the knot should be enforced exactly, and be visually apparent
  \end{description}
}


P3, line 31: Define ``visual continuity".  Are you referring to tangent
continuity, $G1-$, or $G2-$ continuity?

P3, line 33:  The ``understrand" has not been designated yet.

\textcolor{blue}{fixed}%issue#23

Figure 2:  Specify what order Bezier curves are being used.

\textcolor{blue}{fixed}%issue#24

P4, line 18: A gentle introduction to ``R" would help to make the
following easier to understand.

P4, lines 27-32: What is the meaning of this? - Six points defining
two handle pairs?  (Why the extreme precision with 8 digits?)

\textcolor{blue}{Only the first six lines are shown for brevity, I have
  changed the text accordingly: ``Above we see only the first six lines
of the object"}


Figure 3: What is the color assignment? - one color for each Bezier
segment?  How many circles are drawn for each segment?  (Say that
solid blotches result from densely overlapping circles.)

\textcolor{blue}{The original caption was lacking detail.
  \begin{description}
  \item[OLD text] ``The \emph{path} of (unoptimized) knot~$7_6$,
    showing Bezier handles as thin straight lines and circles.  The
    coloured circles have a radius proportional to the curvature (that
    is, the reciprocal of the radius of curvature) along the strand;
    note large curvature at loop on left"
  \item[NEW text] ``The \emph{path} of (unoptimized) knot~$7_6$,
    showing Bezier handles as thin straight lines and circles.  The
    coloured circles have a radius proportional to the curvature (that
    is, the reciprocal of the radius of curvature) along the strand.
    Colouring is arbitrary, one color for each Bezier segment; solid
    blotches result from densely overlapping circles.  Note large
    curvature at loop on left"
  \end{description}
  Each segment has one circle per control point (which defaults to
  $n=100$) but this is user-settable.
}



Figure 4: Why is this required?  -- Does the user have to specify the
type of crossing via an "overunder" object?  -- (Lines 53-62?)

P10, line 17: ``... a vertical line of symmetry." - Better to say that
this figure has $D_5$ symmetry with five mirror lines.


\textcolor{blue}{Done.}

P11, line 26: It seems that all the functionality needed to deal with
links is already present in the software packages mentioned, as long
as one can let the program go around more than one single closed loop.

\textcolor{blue}{I have spent many many hours pondering how to deal
  with links.  As the referee says, much of the required functionality
  is (in principle) already present in the software.  Each component
  would have its own badness, and in addition there would be
  $n(n-1)/2$ inter-component interaction terms measuring features such
  as closeness between non-intersecting strands of different
  components.  Several problems stood out: firstly, the problem of
  inter-component points.  Currently this is a (symmetrical) Boolean
  matrix of strands with rows and columns corresponding to strands,
  and entries corresponding to whether that strand pair intersects or
  not.  The link generalization would be a symmetric matrix of
  intersection matrices: an object with four indices and entry
  (i,j,k,l) indicating whether strand j of component i intersects
  strand k of component l.  I found this object to be unweildy to work
  with and difficult to manipulate.  The second problem was the
  imposition of symmetry.  Sometimes one might desire that some
  components of a link have a particular kind of symmetry and others
  to have a different kind (or no!) symmetry.  One unexpectedly
  difficult problem was that one might have a subset of links that
  individually possess no symmetry but collectively possess mirror-,
  rotational-, or indeed dihedral- symmetry.  Thistlethwaite's L4a1 is
  problematic: one would expect the two components to be identical
  except for a 90-degree rotation, and implementing this in the
  context of the package I found impossible. L6a3 is similarly
  difficult.  We would need new badnesses too.  Consider L6a4; the
  smaller component should be not only as circular as possible (?),
  but also one might desire the overall length to be smaller than that
  of the other component.\\ My overall view was that dealing with
  links would be possible in principle but, due to a number of rather
  poorly delineated reasons (which are not really suitable for
  inclusion in a scholarly publication), an order of magnitude harder
  than the single-component links considered in the submission.  I'm
  not saying it's impossible, but just very difficult and, for me,
  firmly in the category of ``further work''.  I hope that's OK.}

Fig. 10: Why is the absence of mirror symmetry pointed out here?

Fig. 11: Why is this table shown?  It is not mentioned in the main
text.  The figures are ``better" than in Figure 1, but still not
optimal in the spirit of this paper.  ``Rolfsen" needs a reference.


\textcolor{blue}{Rolfsen now cited.}

References to some of the background material are somewhat random.
E.g.  For JMA readers, ``The Knot Book" by Colin Adams would be a good
introductory reference.

Ref [10]  Give a more up-to-date access date.

Ref [11]  Give URL of  Inkscape:  {\tt https://inkscape.org/}

Ref [14]  Give URL of  R Core Team:  {\tt https://www.r-project.org/}


\textcolor{blue}{All amended as suggested}

\section*{Detailed rebuttal: Reviewer \#2}

This lovely article addresses the interesting question of
how to produce pleasing and informative images of two dimensional knot
diagrams. In addition to the discussion it describes how to use a
combination of the software packages Inkscape and R to optimise knots.

At the moment, however the paper is a little too close to simply being
a software manual for the software, and needs in particular to
describe the techniques used by the software to create the optimised
images.

In addition there is a rather consistent use of absolute terms for
pleasing vs ugly knots. Aesthetics is a complex field, but generally
does not provide strong answers. While the choices made in this paper
are reasonable, they are simply stated, rather than including a
justification for why they are chosen. They certainly only apply to a
particular knid of knot diagram. This can be see in the final image
showing all the knots with up to 8 crossings. While some of the
resulting diagrams are beautiful, I particularly enjoyed {\tt 8\_10}
and {\tt 8\_17}, some are not so pleasing to me. In particular the
algorithm given likes to over-tighten when the line wraps round
itself. This can be seen particularly in {\tt 8\_3}, but also in {\tt
  8\_5}, {\tt 8\_6} and {\tt 8\_11}. This is itself a personal
preference not an absolute, but a discussion of how to adjust the
set-up of the software to make personal aesthetic choices would
strengthen the paper.

\textcolor{blue}{This is indeed an insightful comment.  My original
  view was that there was a single set of weightings for the various
  badnesses, that most people would agree upon; and my task as
  software engineer was to find the projection that optimizes these
  standard weightings.\\ \\ Following this comment, I realise with a
  shock that my weightings do not reflect some universal ideal.
  Rather, my choice of weightings are as subjective as any other
  aspect of the software and, as such, are peculiar to me.  This is
  obvious in retrospect but was not clear to me at all until just
  now.\\ \\ One of the benefits of the {\tt knotR} software is that it
  allows this kind of conversation to occur; I have added a discussion
  to the manuscript, just before the conclusions section.  There, I
  present projections of $8_3$ and $8_5$ but with different penalties
  for angle crossing terms.  We see the results and can compare like
  with like, perhaps coming to a deeper understanding of pleasingness
  of projections in the process.\\ \\}

The paper also needs to do a far stronger job of placing itself within
the relevant literature. Simply showing that a single diagram from
wikipedia has weaknesses is not sufficient. The nineteenth century
series of papers ``On Knots" by Peter Guthrie Tait in the Transactions
of the Royal Society of Edinburgh contain far superior images for
example. This is not the first paper to look at making pleasing knot
diagrams, the software knotplot for example offers many tools to do a
similar optimisation to the one discussed here, but is not cited,
Laura Taalman and Colin Abrams among others have multiple papers
addressing the creation of different knot representations and
optimising their form. More broadly the idea of using optimisation in
mathematical art has been used widely, the work of Robert Bosch and
Craig Kaplan springing immediately to mind.

For these reasons the current paper is not really strong enough for
publication. In order to get strong enough I would propose three major
pieces of work:

1) Conduct a thorough literature review of work on the creation of
knot diagrams and their aesthetics and place this paper into that
context as well as the broader context of optimisation in art. % issue~#39

\textcolor{blue}{I have added a literature review as suggested that
  discusses knot diagrams from early Celtic artwork through to
  mathematica.\\ \\ The referee makes the perfectly reasonable
  suggestion of discussing the broader context of optimization of art.
  I will address this comment colloquially.  It is obvious to me, and
  clearly to the referee as well, that numerical optimization
  techniques are well-suited to ``rounding off rough edges'' from a
  wide range of applications; and further that numerical optimization
  will produce ``nice looking'' results.  Is is also reasonable to
  assume that such techniques will have been used in the past by
  graphical artists.  But I can find no evidence of such activity,
  after an extensive online search and consulting with two librarians
  (mathematics and art subject specialists).  Perhaps we are using the
  wrong search terms?  I would gladly take suggestions on previous
  work to cite.  I have added a sentence to the effect that this seems
  to be new to the ``further work'' section.}
  

2) Include a discussion of why the aesthetic choices made for the
optimisation are pleasing as well as how to adjust the methodology in
order to make different personal choices. It might even be interesting
to develop some distinct styles of diagrams (by changing the ``badness"
function, and use the methods to create optimised images for each
style.

3) Give basic information about the specific methods being used for
the optimisation rather than just using the function given in R as a
black box.

With these changes the paper could be an interesting one, discussing
the context of knot diagrams more broadly and providing practical help
to help create sets for oneself.


\end{document}
