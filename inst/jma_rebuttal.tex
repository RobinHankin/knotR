\documentclass[12pt]{article}
\usepackage{xcolor}
\begin{document}

\section*{Rebuttal to second editorial review: overall response}

(Below, the reviewer's comments are in black, and my replies to the
issues are in \textcolor{blue}{blue}.  I have indicated changes to the
manuscript where appropriate.

Short story: I have accommodated all the comments with rewording.  The
resulting document is, I believe, stronger and more scholarly than
before and I recommend it to you.  


\section*{Detailed rebuttal: Reviewer \#1}

This short article is a tutorial example of how to use Inkscape and
KnotR to beautify 2-dimensional knot diagrams found in the knot
tables.  The cost function used in optimizing a given diagram tries to
minimize the overall bending energy of the knot curve, preserve all
shown 2D symmetries, and render all strand crossings as close to 90
degrees as possible.  The paper does not discuss the underlying
mathematics by which this optimization is achieved, but relies on the
power built into the software packages mentioned.  Also, the user has
to characterize explicitly the desired symmetries; there is no ``magic"
power that discovers the maximal possible symmetry that a given knot
can assume.  This exemplified by Figure 7, where the Figure-8-knot is
drawn with a single mirror axis, even though this same knot can
readily be drawn with two mirror lines; -- and also in Figure 9 where
a symmetrical and an asymmetrical rendering of the same knot is
presented.

\textcolor{blue}{This is a good characterisation of the submission,
  but the comment that there is no ``magic" power ascertaining
  maximal possible symmetry that a given knot may assume is a good
  one, and one that I had not explicitly realised.  I have added a
  brief discussion of this issue to the caption of Figure 9 that
  discusses the Perko pair:
  \begin{description}
    \item[OLD text] ``Two representations of knot~$10_{125}$, known as the 
     Perko Pair"
    \item[NEW text] ``Two representations of knot~$10_{125}$, known as the 
      Perko Pair.   The software requires the user to specify the symmetry
      (mirror or rotational) of a knot projecton and has no notion of
      topological invariance of a knot"
      \end{description}
}

This paper is clearly useful to someone who specifically wants to
clean up a given rendering of a particular knot, but it offers little
to readers who are not familiar with Inkscape and KnotR.

Detailed comments about the presentation:

Abstract: It should mention the software packages that are used in
this paper.

\textcolor{blue}{Done.  The abstract is now completely rewritten.}

P1, line 46:  ``Knot theory" -- ?  -- Is this an incomplete reference?

\textcolor{blue}{Corrected, this was a typo}

P2, line 43: ``uniform" curvature is not possible, since many knots
have inflection points.  It is better to ask for 'smoothly changing'
curvature, and to limit maximal curvature.

\textcolor{blue}{Corrected:
  \begin{description}
  \item[OLD text] Curvature to be as uniform as possible
\item[NEW text] Curvature to be as smoothly changing as possible, with
  limited maximal curvature
  \end{description}
}
  


P2, line 46: What does it mean for symmetry to be ``present" in the
knot?

P3, line 31: Define ``visual continuity".  Are you referring to tangent
continuity, $G1-$, or $G2-$ continuity?

P3, line 33:  The ``understrand" has not been designated yet.

Figure 2:  Specify what order Bezier curves are being used.

P4, line 18: A gentle introduction to ``R" would help to make the
following easier to understand.

P4, lines 27-32: What is the meaning of this? - Six points defining
two handle pairs?  (Why the extreme precision with 8 digits?)

Figure 3: What is the color assignment? - one color for each Bezier
segment?  How many circles are drawn for each segment?  (Say that
solid blotches result from densely overlapping circles.)

Figure 4: Why is this required?  -- Does the user have to specify the
type of crossing via an "overunder" object?  -- (Lines 53-62?)

P10, line 17: ``... a vertical line of symmetry." - Better to say that
this figure has $D_5$ symmetry with five mirror lines.

P11, line 26: It seems that all the functionality needed to deal with
links is already present in the software packages mentioned, as long
as one can let the program go around more than one single closed loop.

Fig. 10: Why is the absence of mirror symmetry pointed out here?

Fig. 11: Why is this table shown?  It is not mentioned in the main
text.  The figures are ``better" than in Figure 1, but still not
optimal in the spirit of this paper.  ``Rolfsen" needs a reference.

References to some of the background material are somewhat random.
E.g.  For JMA readers, ``The Knot Book" by Colin Adams would be a good
introductory reference.

Ref [10]  Give a more up-to-date access date.

Ref [11]  Give URL of  Inkscape:  {\tt https://inkscape.org/}

Ref [14]  Give URL of  R Core Team:  {\tt https://www.r-project.org/}


\section*{Detailed rebuttal: Reviewer \#2}

This lovely article addresses the interesting question of
how to produce pleasing and informative images of two dimensional knot
diagrams. In addition to the discussion it describes how to use a
combination of the software packages Inkscape and R to optimise knots.

At the moment, however the paper is a little too close to simply being
a software manual for the software, and needs in particular to
describe the techniques used by the software to create the optimised
images.

In addition there is a rather consistent use of absolute terms for
pleasing vs ugly knots. Aesthetics is a complex field, but generally
does not provide strong answers. While the choices made in this paper
are reasonable, they are simply stated, rather than including a
justification for why they are chosen. They certainly only apply to a
particular knid of knot diagram. This can be see in the final image
showing all the knots with up to 8 crossings. While some of the
resulting diagrams are beautiful, I particularly enjoyed {\tt 8\_10}
and {\tt 8\_17}, some are not so pleasing to me. In particular the
algorithm given likes to over-tighten when the line wraps round
itself. This can be seen particularly in {\tt 8\_3}, but also in {\tt
  8\_5}, {\tt 8\_6} and {\tt 8\_11}. This is itself a personal
preference not an absolute, but a discussion of how to adjust the
set-up of the software to make personal aesthetic choices would
strengthen the paper.

The paper also needs to do a far stronger job of placing itself within
the relevant literature. Simply showing that a single diagram from
wikipedia has weaknesses is not sufficient. The nineteenth century
series of papers ``On Knots" by Peter Guthrie Tait in the Transactions
of the Royal Society of Edinburgh contain far superior images for
example. This is not the first paper to look at making pleasing knot
diagrams, the software knotplot for example offers many tools to do a
similar optimisation to the one discussed here, but is not cited,
Laura Taalman and Colin Abrams among others have multiple papers
addressing the creation of different knot representations and
optimising their form. More broadly the idea of using optimisation in
mathematical art has been used widely, the work of Robert Bosch and
Craig Kaplan springing immediately to mind.

For these reasons the current paper is not really strong enough for
publication. In order to get strong enough I would propose three major
pieces of work:

1) Conduct a thorough literature review of work on the creation of
knot diagrams and their aesthetics and place this paper into that
context as well as the broader context of optimisation in art.

2) Include a discussion of why the aesthetic choices made for the
optimisation are pleasing as well as how to adjust the methodology in
order to make different personal choices. It might even be interesting
to develop some distinct styles of diagrams (by changing the ``badness"
function, and use the methods to create optimised images for each
style.

3) Give basic information about the specific methods being used for
the optimisation rather than just using the function given in R as a
black box.

With these changes the paper could be an interesting one, discussing
the context of knot diagrams more broadly and providing practical help
to help create sets for oneself.


\end{document}
