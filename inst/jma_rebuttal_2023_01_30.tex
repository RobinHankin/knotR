\documentclass[12pt]{article}
\usepackage{xcolor}
\usepackage{amssymb}
\begin{document}

(Below, the reviewer's comments are in black, and my replies to the
issues are in \textcolor{blue}{blue}.  I have indicated changes to the
manuscript where appropriate.

\section*{Rebuttal to "Visually pleasing knot projections" by
  Robin K. S. Hankin (submitted to JMA, manuscript 209604076); review
  30 January 2023}

\textcolor{blue}{ Short story: }



\section*{Detailed rebuttal}



Comments from the Editors and Reviewers:

I appreciate and accept most of the comments made by the author in his
rebuttal and the corresponding fixes in the manuscript.  But I have a
remaining concern that much of this material is perfect for an on-line
tutorial on how to use package [27] to draw optimized knot diagrams,
but JMA is not the right place to publish such a tutorial.

The content that is most fitting for a JMA paper is a discussion of
what makes a knot diagram "beautiful"  and that discussion will
clearly be biased by the personal view of the author (perhaps enhanced
by a "user-study" among family and colleagues).  Thus lines 14 - 20 on
page 4 are of prime importance and could be fleshed out more.

The main contribution to JMA would then be a detailed discussion of
how the author has struggled to capture these goals in the form of
mathematical functions that can be optimized and will then approximate
these goals.  In this respect, the manuscript still has some
shortcomings.

They may be fixed in two ways: Enhance the detailed explanations of
the choices made in creating package [27]; and add a more extended
discussion of a few things that might be approached differently in a
future package ( -- which can also handle links).

In view of this, I recommend that Appendices A and B should be firmly
integrated into the main text.  The discussions therein relate rather
directly to what I see as the main contribution of a JMA paper on this
topic.

A few comments on some rebuttals:

RE: Why use Bezier curves? : A key point, which should be mentioned in
the justification for Bezier curves, is that they interpolate their
end-points. This gives users a convenient, more direct control of
where these curve segments will appear than what they would get from
G2-continuous B-splines, which are only approximating their control
points.

RE: Why not place nodes at the crossing points? : Yes, this will
require a different, slightly more sophisticated drawing interface;
but this may then pay off heavily in a simpler, more efficient
optimization process.

For each crossing point, the user would see an orthogonal "cross-hair"
that can be shifted and rotated as a whole.  In addition, there are
two parameters (handle-pairs) that control the "velocities" of the two
strands through the crossing.  The optimization routine would
fine-tune these velocities to achieve matching curvature for the two
sequentially connected Bezier curves.

And, yes, if a crossing point falls on a line of mirror symmetry, the
crosshair would automatically be angled at 45 degrees with respect to
this line.  And if the knot has n-fold rotational symmetry, n such
cross hairs would be placed at the vertices of a regular n-gon.  (All
this is not such a "nightmare" if it is properly considered from the
beginning.)

I don't expect that anything in the current system should be changed;
but this approach should definitely be considered in a future version
of such a package.


RE: Mismatched diagram and symmetry specifications: "This situation
cannot occur" -- If the user makes a mistake, will the current system
send a suitable error message?

RE: A hierarchical way of dealing with symmetry: "This is exactly what
I have done!  The referee gives an accurate summary of the code's
methodology."  - Place a corresponding, extended description into the
main text.

RE: A slider interface for changing some optimization parameters: It
is clear why this would not be practical in the current system, where
the optimization of a pretzel knot "may take a few weeks."  But this
also suggests that there is much room for improvement!  Perhaps some
badness functions are not really needed or are badly implemented.

RE: Link projections: In a future system, the planning for handling
links should come in right at the beginning.  It is not at all clear
why "Each component would have its own badness."

Many of the important badness functions (e.g. separations of nodes or
of non-crossing curve segments) are based purely on local geometry and
need not know what component of a link they refer to.  Perhaps the
overall length of a knot or of a link-component is not a good metric,
and should be replaced with some other functions of the length of
curve segments between crossings.

Some detailed comments on manuscript R2:

Page 6, line 20: How many (x,y) pairs are there?
\textcolor{blue}{\\Object {\tt k76} has a total of 49 lines and I have
  indicated this in the text.  Old text: ``Above we see only the first
  six lines of the object".  New text: ``The object has 49 lines, of
  which only the first six are shown".\\ \\}

Page 6, Line 36: Why ``$\mathbb{R}^{50}$"? - Are there a total of 50
parameters?  What are they?  \textcolor{blue}{\\This was not clear and
  I have rephased.  Old text: ``The optimization typically proceeds
  over $\mathbb{R}^{50}$\ldots". New text: ``The optimization
  proceeds over $\mathbb{R}^n$ where $n$ is the number of parameters
  needed to specify the path; typically $n\simeq 50$ for the knots
  presented here".\\ \\}

Page 6, Line 39: ``... the overall bending energy." {\tt ->>} 'minimizing
overall bending energy.'
\textcolor{blue}{\\Fixed\\ \\}

Page 6, Line 42:  What is "{\tt > b <}" ?
\\ \\
\textcolor{blue}{This is just a detail of standard R syntax.  The
  first ``{\tt > }" is just the R prompt; letter {\tt b} is just a
  variable; and the symbol ``{\tt <-}" usually has a gloss of
  ``becomes".  Thus {\tt b <- 4} is read ``{\tt b} becomes 4". The
  reason I need a variable is that I want to show the result of
  applying two functions, {\tt total\_crossing\_angles()} and {\tt
    total\_bending\_energy()}, to the same variable {\tt b}.}


Figures 3 and 5: Try running these figures with only 50 points (and
circles) for each Bezier segment.

Figure 4 caption: Much of the explanation of "7 3" versus "3 7" is
already in the main text.

Page 8, Line 60: "Why 64 real variables"?  What are they?  How does
this agree with "R50" on page 6 ?

Figure 7: Where does the spiky artifact in the upper right lobe come
from?  \textcolor{blue}{\\I see nothing in figure 7 that could be so
  described.  I expect this is a rendering issue, if so it will be
  sorted out at proofreading stage.\\ \\}


Page 12, Line 52: ``...a rotationally symmetric" ->> ``... a knot with
pure cyclic rotational symmetry, rather than dihedral symmetry"
  \textcolor{blue}{Done\\ \\}




Figure 9: Why did the user choose to place three nodes on the outer
lobes, rather than only two? - Was this easier to express the desired
symmetry??  \textcolor{blue}{Sort of.  Having more nodes allows for
  finer and more sophisticated control over the path, and having too
  few nodes results in a displeasing, unnatural flabby configuration.
  On the other hand, having too many nodes results in the optimization
  routine thrashing and failing to converge.  In this case it is
  desirable to have an odd number of nodes: 12,4,16,8,20 are
  constrained by the symmetry routine to be exactly on one of the five
  lines of symmetry, and to intersect its line at exactly $90^\circ$.
  Frankly the choice of number of nodes has been troublesome in the
  development of the software.  Perhaps this issue is of interest to
  the JMA readership and I have added a brief sentence discussing it
  to the manuscript: ``Note that increasing the number of nodes
  provides more control over the path at the expense of a more
  difficult optimization process".  Also, at the discussion of Figure
  9: ``from Figure 9 we see that these nodes are constrained to lie on
  one of the five lines of symmetry, cutting it at $90^\circ$".}

Page 16, Line 47: "The strands exert a couple but no force on one
another." ??  Lengthy discussions should be relegated to the main
text, rather than be in figure captions.

Figure 14: All the knots are nicer than the ones in Ref [24].  Good
results!

References: Several of them are incomplete and need an URL or a more
detailed description where the source can be found.




\end{document}
